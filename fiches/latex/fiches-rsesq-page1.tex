% -----------------------------------------------------------------------------
% Copyright © Institut National de la Recherche Scientifique (INRS)
% https://github.com/cgq-qgc/rsesq-bulletin
%
% Created by Jean-Sébastien Gosselin
% jean-sebastien.gosselin@inrs.ca
%
% This work is licensed under the terms of the CC BY 4.0 License as published
% by the Creative Commons nonprofit organization. For more details, see the
% CC BY 4.0 License at https://creativecommons.org/licenses/by/4.0/.
% -----------------------------------------------------------------------------

% Informations sur la station de suivi, localisation, photo et notes

% Idéalement, la photo du puits devrait avoir un ratio d'environ 1.33 et avoir
% des dimensions s'approchant de 1500 x 1125.

\newpage
\noindent
\begin{tcbposter}[
  poster = {
    % showframe,
    columns = 2,
    height= \textheight,
    width = \textwidth, 
    spacing=2em},
  boxes = {colback=white, fonttitle=\boxtitlefont, boxrule=\boxrule,
           colframe=blue, arc=0pt, colbacktitle=blue, coltitle=white}
  ]
  \posterbox[adjusted title=Informations sur la station de suivi]
    {name=info, column=1, span=2, below=top}
    {\fontsize{12}{14}\bodyfont{
    \begin{tabularx}{\textwidth}{%
      @{}>{\raggedright\arraybackslash}X@{}@{}l@{}}
    Municipalité (code) : \munname{} (\muncode) &
    Latitude : \latitude
    \\
    MRC : \mrc{} &
    Longitude : \longitude
    \\
    Région : \region{} &
    Altitude : \altitude{}
    \\
    \addlinespace[1em]
    Nappe : \typenappe &
    Système : NAD83
    \\
    Type d’aquifère de l’intervalle crépiné : \typeaquicrepine{} &
    Précision : \geosysprec
    \\
    \multicolumn{2}{@{}l@{}}{Type d’aquifère du niveau d'eau : 
    \typeaquinappe{}}
    \\
    \addlinespace[1em]
    \multicolumn{2}{@{}l@{}}{Période d’opération : \operationperiod}
    \\
    \end{tabularx}
    \begin{tabularx}{\textwidth}{@{}l@{}@{}X@{}}
    Lacunes dans les données :~ & \gapsdata \\
    \end{tabularx}
    }}
  \posterbox[adjusted title=Localisation, valign=center, halign=center]
    {name=localisation, column=1, row=2, between=info and bottom, span=1.1}
    {\includegraphics[width=\linewidth]{\pathlocal}}
  \posterbox[adjusted title=Photo, valign=center, halign=center]
    {name=photo, column*=2, below=info, span=0.9}
    {\includegraphics[width=\linewidth]{\pathphoto}}
  \posterbox[adjusted title=Notes]
    {name=note, column*=2, between=photo and bottom, span=0.9}
    {\fontsize{10}{12}\bodyfont{\notes{}}}
\end{tcbposter}
\clearpage

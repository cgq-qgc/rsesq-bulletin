\documentclass[]{fiches-rsesq}

\usepackage{afterpage}
\usepackage{pdflscape} % For putting a page in landscape format.
\usepackage{rotating}

\usepackage{graphicx} 
\usepackage{tabularx}
\usepackage{microtype} % Nicer spacing between words
\usepackage[most]{tcolorbox}
\usepackage[export]{adjustbox}  % For the "valign" and "trim" options in
                                % \includegraphics


%==============================================================================
% VARIABLES
%==============================================================================
\newcommand{\stationid}{023440005}
\newcommand{\municipality}{Sainte-Justine}
\newcommand{\mrc}{Les Etchemins}
\newcommand{\region}{Chaudière-Appalaches}
\newcommand{\notes}{Notes en lien avec la station piézométrique ou les 
données.}

%===============================================================================
% PATHS
%===============================================================================
\newcommand{\pathlocal}{./cartes_puits_rsesq/Cartes_Puits_RSESQ_02G47001}
\newcommand{\pathphoto}{./photos_puits_rsesq/photo_puit_P19_scaled}
\newcommand{\pathschem}{./schema_puits_rsesq/schema_puits_03030008.pdf}
\newcommand{\pathhstat}{./hydrostat_puits_rsesq/hydrostat_puits_p19}

\createfooterheader{\stationid}{\municipality}


\begin{document}


%==============================================================================
% PAGE 1
%==============================================================================

\noindent
\begin{tcbposter}[
    poster = {
        % showframe,
        columns = 2,
        height= \textheight,
        width = \textwidth, 
        spacing=2em},
    boxes = {colback=white, fonttitle=\boxtitlefont, boxrule=\boxrule,
             colframe=blue, arc=0pt, colbacktitle=blue, coltitle=white}
    ]
    \posterbox[adjusted title=Information sur la station de suivi]
        {name=info, column=1, span=2, below=top}
        {
        \fontsize{12}{14}\bodyfont{
        \begin{tabularx}{\textwidth}{
            @{}
            >{\raggedright\arraybackslash}X
            l
            @{}}
        Municipalité (code) : \municipality{} (67020) & Latitude : 46.40819 \\
        MRC : \mrc{} & Longitude : -70.3709 \\
        Région : \region{} & NAD : 83\\
        & Élévation du sol : 401.0 m NMM\\
        \multicolumn{2}{@{}X@{}}{Nappe : libre et non influencée}\\
        \multicolumn{2}{@{}X@{}}{Type d’aquifère de l’intervalle crépiné : roc}\\
        \multicolumn{2}{@{}X@{}}{Type d’aquifère de la nappe : granulaire}\\
        \\
        \multicolumn{2}{@{}X@{}}{Période d’opération : date – aujourd'hui}\\
        \multicolumn{2}{@{}X@{}}{Lacunes dans les données : 1973 à 1985, 
        1995, 2014}\\
        \end{tabularx}}
        }
    \posterbox[adjusted title=Localisation, valign=center, halign=center]
        {name=localisation, column=1, row=2, between=info and bottom, span=1.1}
        {\includegraphics[width=\linewidth]{\pathlocal}}
    \posterbox[adjusted title=Photo, valign=center, halign=center]
        {name=photo, column*=2, below=info, span=0.9}
        {\includegraphics[width=\linewidth]{\pathphoto}}
    \posterbox[adjusted title=Notes]
        {name=note, column*=2, between=photo and bottom, span=0.9}
        {
        \fontsize{10}{12}\bodyfont{\notes{}}
        }
\end{tcbposter}


%===============================================================================
% PAGE 2
%===============================================================================
\newpage
\noindent
\begin{tcbposter}[
    poster = {
        % showframe,
        columns=1,
        rows=1,
        height=\textheight,
        width=\textwidth},
    boxes = {colback=white, boxrule=0mm, fonttitle=\fontsize{14}{16}\boldfont,
             colframe=white, coltitle=black, left*=0pt, right*=0pt,
             bottom=0pt}
    ]
    \posterbox[adjusted title=Schéma de la station de suivi,
               valign=center, halign=center, center title]
        {name=context, column=1, between=top and bottom}
        {
        \includegraphics[trim={1.7cm 3.5cm 2.75cm 5.25cm}, clip, height=0.95\textheight]{\pathschem}
        }
\end{tcbposter}


%===============================================================================
% PAGE 3
%===============================================================================
\newpage

%===============================================================================
% PAGE 4
%===============================================================================
\newpage

%===============================================================================
% PAGE 5
%===============================================================================
\newpage
\noindent
\begin{tcbposter}[
poster = {
    showframe,
    columns = 1,
    rows = 1,
    height= \textheight,
    width = \textwidth},
boxes = {colback=white, boxrule=0mm, fonttitle=\fontsize{14}{16}\boldfont,
         colframe=white, coltitle=black, boxsep=0pt}
]
\posterbox[adjusted title=Contexte de la station  de suivi,
           valign=center, halign=center, center title, left*=0pt, right*=0pt,
           bottom=0pt]
    {name=context, column=1, between=top and bottom}
    {
    \includegraphics[width=\linewidth]
    {./contexte_puits_rsesq/Bulletin_Matrices_Fiche_Signaletique_02340001}
    }
\end{tcbposter}

\end{document}